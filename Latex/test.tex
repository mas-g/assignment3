\documentclass[a4paper,12pt]{article}
\begin{document}


\begin{Huge}
\begin{center}
\begin{normalsize}

\textbf{MAKERERE UNIVERSITY } \\
\textbf{FACULTY OF COMPUTING AND INFORMATICS TECHNOLOGY} \\
\textbf{DEPARTMENT OF COMPUTER SCIENCE} \\
\textbf{BACHELOR OF SCIENCE IN COMPUTER SCIENCE} \\
\textbf{BIT 2207 RESEARCH METHODOLOGY} \\
\textbf{YEAR 2} \\


\textbf{\sc MASIGA DAVID KELVIN } \\
\textbf{\sc Reg No: 16/U/579 } \\
\textbf{\sc std No: 216000507}\\
\end{normalsize}
\end{center}
\end{Huge}
\newpage

\title{THE LEVEL OF YOUTH UNEMPLOYMENT IN UGANDA}
\maketitle    
\section{Introduction}
The document below describes the research techniques that were used as per the research carried out on the level of youth unemployment in Uganda. This very document contains sample data that was collected using forms to collect data.  
\section{Abstract}                                                                                                                                                                                    Uganda is one of the countries in the world with the fastest growing population of which the youths take a greater percentage. Due to this rapid growing population, this has increased on the level of unemployment in the country.
Unemployment Rate in Uganda increased to 2.28 percent in 2016 from 2.15 percent in 2015. Unemployment Rate in Uganda averaged 2.41 percent from 1991 until 2016, reaching an all time high of 3.50 percent in 2002 and a record low of 0.94 percent in 1991.According to Wikipedia The unemployment rate for young people ages 15-24 years is 83 percent.This rate is even higher for those who have for those who have formal degrees and live in the urban area.	 
The working age ranges from around 15-64 years. A research carried out in 2012 showed that Uganda's national unemployment rate stood at 11.7 percent. In 2010 according to a research that was carried, it showed that the rural unemployment rate was at 1.7 percent, the urban unemployment rate was at 12 percent and the unemployment rate was at 32.2 percent.
\section{Data collection form}
In this research, one of the methods that was used to carry out the investigation included questionnaire where forms were made to collect specific information and these forms were moved from home to home to be filled by different in different districts and locations all over the country.
The following fields describe the data that was collected from the forms:
\subsection{The name of the person:}
This field describes the name of the person filling in the form.
\subsection{The gender:}
This field describes the sex of the person whether male or female.

\subsection{The age:}
This field contains the age of the person.
\subsection{District of residence:}
This contains the district name in which the person resides.
\subsection{Location:}
This contains the GPS co-ordinates describing the location of the person.
\subsection{Marriage Status:}
This field enables us to know the person is married or not.
\subsection{Employed/Unemployed:}
This field enables us to know whether the person is employed or not.
\subsection{The picture.}
This field contains an image that was taken at the home were information has been gathered showing the current standards of living.

Below is a table showing a sample of data that was collected:

\begin{center}
\begin{tabular}{|c|c|c|c|c|}
\hline
Name & Gender & Age & District & Employed/not  \\ [0.5ex]
\hline
masiga & male & 26 & busia & yes  \\ [0.5ex]
\hline
fella & female & 18 & arua & no  \\ [0.5ex]
\hline
eddie & female & 20 & lira & no  \\ [0.5ex]
\hline
charles & male & 33 & luweero & yes  \\ [0.5ex]
\hline
semate & female & 24 & masaka & no  \\ [0.5ex]
\hline
\end{tabular}
\end{center}`

\end{document}